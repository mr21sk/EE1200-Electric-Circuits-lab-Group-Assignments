%% Signed and Unsigned Binary Numbers Notes
\documentclass[a4paper,12pt]{article}
\usepackage{amsmath, amssymb}
\usepackage{graphicx}
\usepackage{enumitem}
\usepackage{fancyhdr}

\pagestyle{fancy}


\begin{document}

\title{\textbf{{1.6} Signed Binary Numbers}}
\author{}
\date{}
\maketitle

\section*{1. Representation of Signed and Unsigned Binary Numbers}

\begin{itemize}
    \item \textbf{Unsigned Numbers:} Represent only positive integers (including zero). All bits are used to represent the magnitude.
    \item \textbf{Signed Numbers:} Represent both positive and negative integers.
    \begin{itemize}
        \item The \textbf{leftmost bit (MSB)} represents the sign:
        \begin{itemize}
            \item $0$ indicates a positive number.
            \item $1$ indicates a negative number.
        \end{itemize}
    \end{itemize}
    \item \textbf{Example:}
    \begin{itemize}
        \item $01001$ (Unsigned) $= 9$ \quad (Signed) $= +9$
        \item $11001$ (Unsigned) $= 25$ \quad (Signed) $= -9$
    \end{itemize}
\end{itemize}

\section*{2. Signed Number Representations}

\subsection*{(a) Signed-Magnitude Representation}
\begin{itemize}
    \item Leftmost bit is the \textbf{sign bit}.
    \item Remaining bits represent the \textbf{magnitude}.
    \item Example (8 bits):
    \begin{align*}
        +9 & = 00001001 \\
        -9 & = 10001001
    \end{align*}
\end{itemize}

\subsection*{(b) 1's Complement Representation}
\begin{itemize}
    \item Negative numbers are represented by \textbf{flipping all bits} of the positive number.
    \item Example:
    \begin{align*}
        +9 & = 00001001 \\
        -9 & = 11110110
    \end{align*}
\end{itemize}

\subsection*{(c) 2's Complement Representation}
\begin{itemize}
    \item Obtain by adding $1$ to the 1's complement.
    \item Simplifies arithmetic operations.
    \item Example:
    \begin{align*}
        +9 & = 00001001 \\
        -9 & = 11110111
    \end{align*}
\end{itemize}

\section*{3. Arithmetic Operations}

\subsection*{(a) Addition in 2's Complement}
\begin{itemize}
    \item Add numbers including the sign bit.
    \item Discard carry out of the MSB.
    \item Example:
    \begin{align*}
        (+6) + (-13): & \\
        00000110 + 11110011 = 11111001 \quad (Result = -7)
    \end{align*}
\end{itemize}

\subsection*{(b) Subtraction in 2's Complement}
\begin{itemize}
    \item Take 2's complement of the subtrahend and add it to the minuend.
    \item Discard carry out of the MSB.
    \item Example:
    \begin{align*}
        (-6) - (-13): & \\
        11111010 + 00001101 = 00000111 \quad (Result = +7)
    \end{align*}
\end{itemize}

\section*{4. Overflow Conditions}
\begin{itemize}
    \item Occurs when the result exceeds the range that can be represented with the given number of bits.
    \item Detection Rule:
    \begin{itemize}
        \item Adding two positive numbers gives a negative result.
        \item Adding two negative numbers gives a positive result.
    \end{itemize}
\end{itemize}

\end{document}

\documentclass{article}
\usepackage{amsmath}
\usepackage{amsfonts}

\begin{document}


\title{Section 1.1}
\author{}
\date{}
\maketitle

\section{Role of Digital Systems}
\begin{itemize}
    \item Digital systems are integral to various fields, including communication, business, traffic control, medical applications, weather monitoring, and scientific research.
    \item They are embedded in everyday devices such as digital telephones, televisions, cameras, and handheld touchscreen devices.
\end{itemize}

\section{Binary Representation in Digital Systems}
\begin{itemize}
    \item Digital systems operate on discrete elements of information, represented using \textbf{binary digits (bits)}—0 and 1.
    \item Groups of bits, called \textbf{binary codes}, represent numbers, characters, and symbols.
    \item The decimal number system can be converted into binary, where each number is expressed using a sequence of bits (e.g., $7_{10} = 0111_2$).
\end{itemize}

\section{Core Components of a Digital Computer}
\begin{itemize}
    \item \textbf{Memory Unit}: Stores programs, data, and intermediate results.
    \item \textbf{Central Processing Unit (CPU)}: Executes arithmetic, logic, and data processing operations.
    \item \textbf{Input/Output Devices}: Allow data entry (e.g., keyboards, touchscreens) and result display (e.g., printers, monitors).
    \item \textbf{Communication Unit}: Enables data exchange through networks, such as the Internet.
\end{itemize}

\section{Binary Logic and Digital Circuits}
\begin{itemize}
    \item Digital circuits process binary signals using \textbf{logic gates} (AND, OR, NOT, etc.).
    \item \textbf{Flip-flops} store binary data and are used in memory and sequential circuits.
\end{itemize}

\section{Advantages of Digital Systems}
\begin{itemize}
    \item \textbf{High Speed}: Modern digital circuits perform operations at millions of cycles per second.
    \item \textbf{Programmability}: Many digital devices can be reprogrammed for different tasks, making them versatile.
    \item \textbf{Reliability}: Error-correcting codes ensure accurate data storage and transmission.
    \item \textbf{Cost Efficiency}: Advances in \textbf{integrated circuit (IC) technology} reduce manufacturing costs while increasing performance.
\end{itemize}

\section{Quantization of Data}
\begin{itemize}
    \item Digital systems process both inherently discrete data (e.g., payroll records) and \textbf{quantized} continuous data (e.g., temperature readings).
    \item \textbf{Analog-to-Digital Converters (ADCs)} transform continuous signals into digital form (e.g., in digital cameras).
\end{itemize}

\section{Digital System Design Using HDL}
\begin{itemize}
    \item Modern digital systems are designed using \textbf{Hardware Description Languages (HDLs)}, which describe circuit functionality in textual form.
    \item HDLs enable simulation, verification, and synthesis of digital circuits before fabrication.
    \item Proper HDL-based design ensures efficient and functional digital hardware.
\end{itemize}

\section{Conclusion}
\begin{itemize}
    \item Digital systems form the foundation of modern computing and information processing.
    \item Their ability to represent and manipulate binary data makes them highly efficient and widely used.
    \item Understanding binary representation, logic circuits, and digital design methodologies is essential for working with digital technology.
\end{itemize}

\end{document}


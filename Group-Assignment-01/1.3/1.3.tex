\documentclass{article}
\usepackage{amsmath}
\usepackage{booktabs}

\title{Section 1.3: Number-Base Conversion}
\date{}

\begin{document}
	
	\maketitle

	\subsection*{Key Concepts}
	\begin{itemize}
		\item Two number representations are \textbf{equivalent} if they have the same decimal value (e.g., $(0011)_8$ and $(1001)_2$ both represent 9).
		\item Conversion from base $r$ to decimal involves expanding the number into a power series and summing the terms.
		\item Conversion from decimal to base $r$ requires separating the number into its integer and fractional parts, as each part is converted differently.
	\end{itemize}
	
	\subsection*{Decimal to Base-$r$ Conversion}
	\begin{itemize}
		\item \textbf{Integer Part}: Divide the number by $r$ repeatedly, accumulating remainders. The remainders (in reverse order) form the base-$r$ representation.
		\item \textbf{Fractional Part}: Multiply the fraction by $r$ repeatedly, accumulating integers. The integers form the base-$r$ representation.
	\end{itemize}
	
	\subsection*{Examples}
	\begin{itemize}
		\item \textbf{Decimal to Binary}:
		\begin{itemize}
			\item Convert $(41)_{10}$ to binary:
			\begin{align*}
				41 \div 2 &= 20 \quad \text{remainder } 1 \\
				20 \div 2 &= 10 \quad \text{remainder } 0 \\
				10 \div 2 &= 5 \quad \text{remainder } 0 \\
				5 \div 2 &= 2 \quad \text{remainder } 1 \\
				2 \div 2 &= 1 \quad \text{remainder } 0 \\
				1 \div 2 &= 0 \quad \text{remainder } 1 \\
			\end{align*}
			Result: $(41)_{10} = (101001)_2$.
		\end{itemize}
		\item \textbf{Decimal to Octal}:
		\begin{itemize}
			\item Convert $(153)_{10}$ to octal:
			\begin{align*}
				153 \div 8 &= 19 \quad \text{remainder } 1 \\
				19 \div 8 &= 2 \quad \text{remainder } 3 \\
				2 \div 8 &= 0 \quad \text{remainder } 2 \\
			\end{align*}
			Result: $(153)_{10} = (231)_8$.
		\end{itemize}
		\item \textbf{Decimal Fraction to Binary}:
		\begin{itemize}
			\item Convert $(0.6875)_{10}$ to binary:
			\begin{align*}
				0.6875 \times 2 &= 1.375 \quad \text{integer } 1 \\
				0.375 \times 2 &= 0.75 \quad \text{integer } 0 \\
				0.75 \times 2 &= 1.5 \quad \text{integer } 1 \\
				0.5 \times 2 &= 1.0 \quad \text{integer } 1 \\
			\end{align*}
			Result: $(0.6875)_{10} = (0.1011)_2$.
		\end{itemize}
		\item \textbf{Decimal Fraction to Octal}:
		\begin{itemize}
			\item Convert $(0.513)_{10}$ to octal:
			\begin{align*}
				0.513 \times 8 &= 4.104 \quad \text{integer } 4 \\
				0.104 \times 8 &= 0.832 \quad \text{integer } 0 \\
				0.832 \times 8 &= 6.656 \quad \text{integer } 6 \\
				0.656 \times 8 &= 5.248 \quad \text{integer } 5 \\
				0.248 \times 8 &= 1.984 \quad \text{integer } 1 \\
				0.984 \times 8 &= 7.872 \quad \text{integer } 7 \\
			\end{align*}
			Result: $(0.513)_{10} = (0.406517)_8$.
		\end{itemize}
	\end{itemize}
	
	\subsection*{Combining Integer and Fractional Parts}
	\begin{itemize}
		\item For numbers with both integer and fractional parts, convert each part separately and combine the results.
		\item Example: $(41.6875)_{10} = (101001.1011)_2$.
		\item Example: $(153.513)_{10} = (231.406517)_8$.
	\end{itemize}
	
	\subsection*{Table of Powers of Two}
	\begin{table}[h!]
		\centering
		\begin{tabular}{cc|cc|cc}
			\toprule
			$n$ & $2^n$ & $n$ & $2^n$ & $n$ & $2^n$ \\
			\midrule
			0 & 1 & 8 & 256 & 16 & 65,536 \\
			1 & 2 & 9 & 512 & 17 & 131,072 \\
			2 & 4 & 10 & 1,024 (1K) & 18 & 262,144 \\
			3 & 8 & 11 & 2,048 & 19 & 524,288 \\
			4 & 16 & 12 & 4,096 (4K) & 20 & 1,048,576 (1M) \\
			5 & 32 & 13 & 8,192 & 21 & 2,097,152 \\
			6 & 64 & 14 & 16,384 & 22 & 4,194,304 \\
			7 & 128 & 15 & 32,768 & 23 & 8,388,608 \\
			\bottomrule
		\end{tabular}
		\caption{Powers of Two}
	\end{table}
	
\end{document}
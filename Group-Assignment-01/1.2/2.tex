\documentclass{article}
\usepackage{amsmath}
\usepackage{array}
\usepackage{graphicx}

\begin{document}

\title{1.2}
\author{}
\date{}
\maketitle

\section{Decimal Number System (Base 10)}
A decimal number consists of digits from 0 to 9, with each digit's place value determined by powers of 10.
For example, the number 7392 can be expressed as:
\begin{equation}
  7 \times 10^3 + 3 \times 10^2 + 9 \times 10^1 + 2 \times 10^0
\end{equation}

\section{Number Systems and Radix (Base)}
A number system's radix (or base) determines the set of digits and the positional values of those digits:
\begin{itemize}
    \item \textbf{Decimal System (Base 10):} Uses digits 0-9.
    \item \textbf{Binary System (Base 2):} Uses only two digits: 0 and 1.
    \item \textbf{Octal System (Base 8):} Uses digits 0-7.
    \item \textbf{Hexadecimal System (Base 16):} Uses digits 0-9 and letters A-F.
\end{itemize}

\section{Binary Number System (Base 2)}
In the binary system, each digit is either 0 or 1 and is multiplied by a power of 2.
For example, converting $1100.11_2$ to decimal:
\begin{equation}
  (1 \times 2^3) + (1 \times 2^2) + (0 \times 2^1) + (0 \times 2^0) + (1 \times 2^{-1}) + (1 \times 2^{-2}) = 12.75_{10}
\end{equation}

\section{Octal Number System (Base 8)}
The octal system has eight digits (0 to 7). For example, converting $(127.4)_8$ to decimal:
\begin{equation}
  (1 \times 8^2) + (2 \times 8^1) + (7 \times 8^0) + (4 \times 8^{-1}) = 87.5_{10}
\end{equation}

\section{Hexadecimal Number System (Base 16)}
The hexadecimal system has sixteen symbols (0-9 and A-F). For example, converting $(B65F)_{16}$ to decimal:
\begin{equation}
  (11 \times 16^3) + (6 \times 16^2) + (5 \times 16^1) + (15 \times 16^0) = 46687_{10}
\end{equation}

\section{Powers of Two and Storage in Computers}
The binary system is widely used in computing. Common memory sizes:
\begin{itemize}
    \item $2^{10} = 1024$ bytes = 1 KB
    \item $2^{20} = 1,048,576$ bytes = 1 MB
    \item $2^{30} = 1,073,741,824$ bytes = 1 GB
    \item $2^{40} = 1,099,511,627,776$ bytes = 1 TB
\end{itemize}

\section{Arithmetic Operations in Binary}
Binary arithmetic follows similar principles as decimal arithmetic but uses only 0s and 1s.

\subsection{Binary Addition Example}
\begin{align*}
  &\quad 101011 \
+ &\quad 100111 \\
\hline
  &\quad 1010000
\end{align*}

\subsection{Binary Subtraction Example}
\begin{align*}
  &\quad 101010 \\
- &\quad 100111 \\
\hline
  &\quad 000011
\end{align*}

\subsection{Binary Multiplication Example}
\begin{align*}
  &\quad 1011 \\
\times &\quad 101 \\
\hline
  &\quad 1011 \\
  &\quad 0000 \\
+ &\quad 1011 \\
\hline
  &\quad 110111
\end{align*}

\end{document}


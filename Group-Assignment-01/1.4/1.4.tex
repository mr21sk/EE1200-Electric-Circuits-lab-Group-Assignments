\documentclass{article}
\usepackage{amsmath}
\usepackage{booktabs}

\title{Section 1.4: Octal and Hexadecimal}
\date{}

\begin{document}
	
	\maketitle

	\subsection*{Key Concepts}
	\begin{itemize}
		\item Octal (base-8) and hexadecimal (base-16) systems are widely used in digital systems because they provide a compact representation of binary numbers.
		\item Each octal digit corresponds to \textbf{3 binary digits}, and each hexadecimal digit corresponds to \textbf{4 binary digits}.
		\item Conversion between binary, octal, and hexadecimal is straightforward due to the direct relationship between their bases.
	\end{itemize}
	
	\subsection*{Binary to Octal Conversion}
	\begin{itemize}
		\item Partition the binary number into groups of \textbf{3 digits} (starting from the binary point).
		\item Convert each group to its corresponding octal digit.
		\item Example:
		\[
		(10\,110\,001\,101\,011.\,111\,100\,000\,110)_2 = (26153.7406)_8
		\]
	\end{itemize}
	
	\subsection*{Binary to Hexadecimal Conversion}
	\begin{itemize}
		\item Partition the binary number into groups of \textbf{4 digits} (starting from the binary point).
		\item Convert each group to its corresponding hexadecimal digit.
		\item Example:
		\[
		(10\,1100\,0110\,1011.\,1111\,0010)_2 = (2C6B.F2)_{16}
		\]
	\end{itemize}
	
	\subsection*{Octal/Hexadecimal to Binary Conversion}
	\begin{itemize}
		\item Convert each octal digit to its \textbf{3-digit binary equivalent}.
		\item Convert each hexadecimal digit to its \textbf{4-digit binary equivalent}.
		\item Examples:
		\begin{itemize}
			\item Octal to Binary:
			\[
			(673.124)_8 = (110\,111\,011.\,001\,010\,100)_2
			\]
			\item Hexadecimal to Binary:
			\[
			(306.D)_{16} = (0011\,0000\,0110.\,1101)_2
			\]
		\end{itemize}
	\end{itemize}
	
	\subsection*{Advantages of Octal and Hexadecimal}
	\begin{itemize}
		\item Binary numbers are long and difficult to work with, but octal and hexadecimal provide a compact representation.
		\item Example: The binary number \(111111111111\) (12 digits) can be represented as:
		\begin{itemize}
			\item Octal: \(7777\) (4 digits)
			\item Hexadecimal: \(FFF\) (3 digits)
		\end{itemize}
		\item Hexadecimal is particularly useful for representing bytes (8 bits) with just 2 digits.
	\end{itemize}
	
	\subsection*{Table of Numbers with Different Bases}
	\begin{table}[h!]
		\centering
		\begin{tabular}{cccc}
			\toprule
			\textbf{Decimal (base 10)} & \textbf{Binary (base 2)} & \textbf{Octal (base 8)} & \textbf{Hexadecimal (base 16)} \\
			\midrule
			0 & 0000 & 00 & 0 \\
			1 & 0001 & 01 & 1 \\
			2 & 0010 & 02 & 2 \\
			3 & 0011 & 03 & 3 \\
			4 & 0100 & 04 & 4 \\
			5 & 0101 & 05 & 5 \\
			6 & 0110 & 06 & 6 \\
			7 & 0111 & 07 & 7 \\
			8 & 1000 & 10 & 8 \\
			9 & 1001 & 11 & 9 \\
			10 & 1010 & 12 & A \\
			11 & 1011 & 13 & B \\
			12 & 1100 & 14 & C \\
			13 & 1101 & 15 & D \\
			14 & 1110 & 16 & E \\
			15 & 1111 & 17 & F \\
			\bottomrule
		\end{tabular}
		\caption{Numbers with Different Bases}
	\end{table}
	
\end{document}
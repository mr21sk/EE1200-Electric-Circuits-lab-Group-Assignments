\documentclass{article}
\usepackage{amsmath}
\usepackage{amsfonts}

\begin{document}


\title{Section 1.5}
\author{}
\date{}
\maketitle

\section{Complements of Numbers}
\begin{itemize}
    \item Complements are used in digital computers to simplify subtraction operation and for logical manipulation
    \item There are two types of complment for base-r system
    \begin{enumerate}
        \item Diminished radix complement-(r-1)'s complement
        \item Radix complement-r's Complement
    \end{enumerate}
\end{itemize}
\subsection{Diminished Radix Complement}
\begin{itemize}
    \item Given a number N in base r having n digits, the (r-1)'s complement of N is its diminished radix complement , is defined as $(r^n-1)-N$.
    \item For example in decimal system,
    \begin{align}
        \text{The 9's complement of 546700 is} \ 999999- 546700= 453299.\\
        \text {The 9's complement of 012398 is} \ 999999- 012398= 987601.
    \end{align}
    From the above example it is clear that 9's complement can be obtained by subtracting each digit with 9 
    \item In binary
    \begin{align}
        \text{The 1's complement of 1011000 is 0100111.}\\
        \text{The 1's complement of 0101101 is 1010010.}
    \end{align}
    From the above example it is clear that the 1’s complement of a binary number is formed by changing 1’s to 0’s and 0’s to 1’s. 
    \item Generally, in any base-r system (r-1)'s complement is obtained by subtracting each digit in (r-1)
\end{itemize}
\subsection{Radix Complement}
\begin{itemize}
    \item The r’s complement of an n-digit number N in base r is defined as $r^n- N$ for N $\neq$ 0 and as 0 for N= 0.
    \item Notice, r's complement = (r-1)'s complement + 1. i,e
    \begin{align}
        r^n-N=(r^n-1)-N+1
    \end{align}
    \item For example, in decimal system 
    \begin{align}
        \text{the 10's complement of 012398 is 987602}\\
        \text{the 10's complement of 246700 is 753300}
    \end{align}
    \item In binary
    \begin{align}
        \text{the 2's complement of 1101100 is 0010100}\\
        \text{the 2's complement of 0110111 is 1001001}
    \end{align}
    \item  The original number N contains a radix point, the point should be removed temporarily in order to form the r’s or (r- 1)>s complement. The radix point is then
restored to the complemented number in the same relative position.
\end{itemize}
\subsection{Subtraction with complement}
\begin{itemize}
    \item When we subtract borrow carry when the minuend is smaller than the subtrahend. This works when using pen and paper but very inefficient than using complements.
    \item The subtraction of two n-digit unsigned numbers M- N in base r can be done as follows:
    \begin{enumerate}
        \item Add the minuend M to the r’s complement of the subtrahend N. Mathematically,$M+(r^n-N)=M-N+r^n$
        \item If $M \geq N$, the sum the sum will produce an end carry $r^n$
, which can be discarded; what is
left is the result M- N.
        \item if $M<N$, the sum does not produce an end carry and is equal to $r^n$
- (N- M),
which is the r’s complement of (N- M). To obtain the answer in a familiar form,
take the r’s complement of the sum and place a negative sign in front.
    \end{enumerate}
\end{itemize}
\end{document}

